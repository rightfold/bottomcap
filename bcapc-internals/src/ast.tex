\zsection0[ast]{Abstract syntax tree}

The abstract syntax tree
is a tree structure
suitable for
further processing.
It is produced
from the source text
in a process called parsing.
In the \zacronym{AST},
expressions may be nested
arbitrarily.
See figure \ref{ast.example}
for an example program
and its corresponding \zacronym{AST}.

\tikzstyle{znode}=[draw, fill=white, rounded rectangle]

\begin{zfigure}[sidebyside]{
    Example source text (left)
    and equivalent \zacronym{AST} (right).
}
    \label{ast.example}

    \begin{lstlisting}[language=Bottomcap]
f (g x)
  (lambda y in h y)
    \end{lstlisting}

    \tcblower

    \begin{center}
        \begin{tikzpicture}[
            every node/.style={znode},
            level distance=0.5in,
            level 1/.style={sibling distance=1.5in},
            level 2/.style={sibling distance=1.0in},
            level 3/.style={sibling distance=0.5in}
        ]
            \node { \strut }
                child {
                    node { \strut }
                    child { node { \strut f } }
                    child {
                        node { \strut }
                        child { node { \strut g } }
                        child { node { \strut x } }
                    }
                }
                child {
                    node { \strut lambda y }
                    child {
                        node { \strut }
                        child { node { \strut h } }
                        child { node { \strut y } }
                    }
                } ;
        \end{tikzpicture}
    \end{center}
\end{zfigure}

The \zacronym{AST}
does not preserve information such as
whitespace,
punctuation,
and redundant keywords.
Syntactic sugar
remains, however.
It is not eliminated
until \hyperref[anf.lower]{lowering}.
