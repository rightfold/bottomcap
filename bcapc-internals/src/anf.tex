\zsection0[anf]{Administrative normal form}

The adminstrative normal form
is suitable for optimization
and code generation.
It is a simplified form
in which only variables
and constants
occur within other expressions,
and everything else
is bound via bindings.
See figure \ref{anf.example.bcap}
for a comparison
between an \zacronym{AST}
and the equivalent \zacronym{ANF}.

\begin{zfigure}[sidebyside]{
    Example \zacronym{AST} (left)
    and equivalent \zacronym{ANF} (right).
}
    \label{anf.example.bcap}

    \begin{lstlisting}[language=Bottomcap]
f (g x)
  (lambda y in h y)
    \end{lstlisting}

    \tcblower

    \begin{lstlisting}[language=Bottomcap, mathescape=true]
let A is g x in
let B is lambda C in
    let D is h C
    in D in
let E is f A in
let F is E B in
E
    \end{lstlisting}
\end{zfigure}

\zsection1[anf.lower]{Lowering}

The translation
from \zacronym{AST}
to \zacronym{ANF}
happens through
the \zacronym{ANF} construction \zacronym{EDSL}.
This process is called lowering.
Lowering proceeds
by traversing the \zacronym{AST}
bottom-up.
During the traversal
a symbol table is kept,
mapping identifiers to symbols.
A symbol is either
a package name,
a global name,
or a local name.
When a variable expression is traversed,
the corresponding symbol is looked up
and appropriate \zacronym{ANF} is generated.
