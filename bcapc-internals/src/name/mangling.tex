\zsection1[name.mangle]{Name mangling}

Name mangling ensures
that two similar names
that are differently qualified
end up as different names
in the translation.
Name mangling also
avoids problems
where targets
have restrictions
on what names may be.
Name mangling
is uniform
across targets.

\zsection2[name.mangle.global]{Global names}

The mangling
of a global name
is each segment normalized
and the concatenation
prefixed with `bcapg'.

\zsection2[name.mangle.local]{Local names}

The mangling
of a local name
in identifier form
is the identifier normalized
and prefixed with `bcapl'.
The mangling
of a compiler-synthesized local name
is the name in decimal
prefixed with `bcaps'.

\zsection2[name.mangle.field]{Field names}

The mangling
of a field name
is the identifier normalized
and prefixed with `bcapf'.

\zsection2[name.mangle.normalize]{Normalization}

The purpose of normalization is
to ensure compatibility
with different target languages
that each have their own restrictions
on what can and cannot be in a name, and
how names collide with each other.
Examples of such restrictions
are case-insensitivity of names
and demanding that all characters
are part of \zacronym{ASCII}.

The normalization process
ensures that a name
consists only of
lower-case \zacronym{ASCII} letters
and \zacronym{ASCII} digits.
To normalize a name,
perform the following steps
one after another:

\begin{enumerate}
\item
    Every instance of `z'
    is prefixed with `z'.
\item
    Every code point that is
    not an \zacronym{ASCII} lowercase letter
    or an \zacronym{ASCII} uppercase letter
    or an \zacronym{ASCII} digit
    is replaced by the eight-digit Unicode scalar value
    of that code point
    in lowercase hexadecimal
    and prefixed with `zs'.
\item
    Every code point that is
    an \zacronym{ASCII} uppercase letter
    is lowercased
    and prefixed with `zu'.
\item
    Prepend the length
    of the normalized name so far
    in decimal
    followed by `z'.
\end{enumerate}
