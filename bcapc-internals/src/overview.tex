\zsection0[overview]{Overview}

bcapc reads
a translation unit
and turns it
into an object.
A translation unit
is an amount of source text%
---possibly from multiple source files---%
that is translated
into a target language.
Cyclic references
between package-level definitions
may exist within a translation unit,
but not between translation units.

After compilation,
objects must be linked
to form an executable.
Linking happens
as a separate invocation
of bcapc,
because it involves
multiple translation units.

Compiling a program with bcapc
is a multi-stage process.
See figure \ref{overview.diagram}
for an overview
of the stages
and how they are connected.

\tikzstyle{zstage}=[draw, fill=white, minimum width=0.75in, minimum height=0.5in]
\tikzstyle{zfinal}=[zstage, dashed]
\tikzstyle{znext}=[draw, ->]
\tikzstyle{zuses}=[draw, dashed]

\begin{zfigure}{The bcapc compilation stages.}
    \label{overview.diagram}
    \begin{center}
        \vspace{4pt}
        \begin{tikzpicture}[node distance=0.25in and 0.75in]
            \node [zstage]                 (parse) { \strut parse }                       ;
            \node [zstage, right=of parse] (lower) { \strut \hyperref[anf.lower]{lower} } ;
            \node [zstage, below=of lower] (check) { \strut check }                       ;
            \node [zstage, right=of lower] (trans) { \strut translate }                   ;
            \node [zfinal, right=of trans] (link)  { \strut link }                        ;

            \path [znext] (parse) -- node [above] { \strut \zacronym{AST} } (lower) ;
            \path [zuses] (lower) --                                        (check) ;
            \path [znext] (lower) -- node [above] { \strut \zacronym{ANF} } (trans) ;
            \path [znext] (trans) -- node [above] { \strut object         } (link)  ;
        \end{tikzpicture}
    \end{center}
\end{zfigure}
